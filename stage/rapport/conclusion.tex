\chapter{Conclusion et perspectives}
Pour conclure ce rapport je souhaite revenir sur quelques points. 

Le problématique du stage était de définir une démarche permettant de savoir si un aéronef présentant des changements de loi de pilotage (modèle hybride) était capable de réussir une mission donnée. Pour répondre à cette problématique nous avions plusieurs objectifs à atteindre, tout d'abord réussir à trouver le formalisme qui allait nous permettre d'intégrer tous nos composants ensemble (dynamique continue, aspect de mission et de l'environnement). Le formalisme qui a été choisi est celui des CSP, il offre une grande flexibilité de représentation et plusieurs solveurs proposent la prise en compte de variables continues et entières dans un même problème.

Par la suite il s'agissait d'écrire le problème de planification dans ce formalisme, pour cela nous avons traduit les aspects de planification classique (variables, actions, objectifs...) dans le formalisme des CSP. La représentation CSP, très mathématique, nous convenait mais nous étions incapables de savoir si elle allait pouvoir être implémentée et surtout si elle allait permettre de résoudre des problèmes à échelle humaine, c'est à dire des problèmes de planification sur plusieurs heures et kilomètres.

Nous avons donc commencé à lister plusieurs outils à notre disposition et les avons testé. JaCoP nous a permis d'aller assez loin d'un point de vue implémentation et résultat, malheureusement le passage à l'échelle a totalement remis en question notre choix. L'utilisation du solveur Cplex semblait être le meilleur choix, il est plus robuste et présente plusieurs fonctionnalités qu'il nous manquait dans JaCoP (variable expression). Le premier choix pour l'utilisation de Cplex a été d'utiliser le langage de programmation OPL, mais ce langage, bien que très proche des mathématiques et donc d'une représentation CSP, ne permettait pas la définition de variable de manière récursive. Nous avons donc rapidement changé pour passer sur la librairie Cplex pour JAVA. C'est cette approche qui nous a permis d'avoir les meilleurs résultats. Tout n'a pas encore pu être testé mais les résultats sont très encourageants.
%Cette librairie est assez longue à prendre en main, c'est pour cela que pour le moment aucun résultat probant n'a pu être produit.

Même si nous n'avons pas encore réussi à tout implémenter, cela ne remet pas en question notre démarche et notre modélisation pour le moment. Ce travail de stage n'est qu'un préambule à la planification de système hybride. Plusieurs pistes de travail nous sont apparues : 

\begin{itemize}
\item[->] Continuer d'écrire la modélisation avec la librairie JAVA de Cplex qui semble être un très bon candidat;
\item[->] Utiliser MatLab et la libraire Cplex pour MatLab : Plutôt que de simuler l'aéronef via le calcul des espaces d'état (donc modèle linéarisé) et de la dynamique de l'automate hybride, il serait peut-être possible de récupérer les sorties du système directement depuis le modèle non-linéaire simulink de l'aéronef. Cela apporterait plus de précision et de réalité vis à vis de l'aéronef. Donc à chaque instant, on donnerait les commandes (variables de décision) en entrée du simulink, et celui ci nous donnerait les sorties (dans notre exemple Altitude, vitesse et angle).
\item[->] Une autre idée est d'utiliser de la recherche locale. Un outil a été développé à l'ONERA : InCell, nous avons commencé à étudier cet outil pour savoir s'il pourrait convenir à notre problème. L'avantage de la recherche locale est que l'on écrit soi-même les heuristiques de recherches de solution, mais cela implique une moins bonne généralisation et automatisation. En effet la recherche locale s'appuie sur des connaissances expertes de la mission, de l'aéronef etc. Donc il n'est pas assuré que nous arriverions à écrire un problème de recherche locale assez général pour pouvoir être appliquée à n'importe quelle mission et/ou aéronef.
\item[->] Une dernière idée est de ne pas tenter de planifier d'un seul coup toute la mission. Un algorithme classique de recherche de chemin dans un graphe pourrait nous fournir des points de passage (non contraint par la dynamique de l'aéronef), puis ensuite nous appliquerons notre démarche entre chaque point de passage (\cite{chanthery_planification_2005}\cite{dicheva_planification_2012}). Cela pourrait permettre d'éviter les problèmes de passage à l'échelle que nous avons eu avec JaCoP.
\end{itemize}