%\phantomsection\addcontentsline{toc}{section}{Références}
\begin{thebibliography}{ABC}	
    \bibitem{kur02} Monika KUROVSKY. \emph{Etude des systèmes dynamiques hybrides par représentation d'état discrète et automate hybride.}. Institut National Polytechnique Grenoble - INPG, 2002.
    \bibitem{cha05} Elodie CHANTERY. \emph{Planification de mission pour un véhicule aérien autonome}. École nationale supérieure de l'aéronautique et de l'espace, 2005.
    \bibitem{art06} Christian ARTIGUES, Dominique FEILLET. \emph{Une méthode exacte pour le problème d'ordonnancement d'atelier avec temps de préparation}. Écoles des Mines d'Alès, 2006.
    \bibitem{her07} Florent HERNANDEZ. \emph{Model-chechking et ordonnancement : Application à la décision de protection phytosanitaire de la vigne}. CEMAGREF, 2007.
    \bibitem{yalmip} Johan LÖFBERG. \emph{A toolbox for modeling and optimization in MATLAB}. Automatic Control Laboratory, 2004.
    \bibitem{goe09} Goebel, R. ; Sanfelice, R.G. et Teel, A. \emph{Hybrid dynamical systems : Robust stability and control for systems that combine continuous-time and discrete-time dynamics}. IEEE Control Systems, 2009.
    \bibitem{boy94} S. BOYD, L. El GHAOUI, E. FERON, V. BALAKRISHNAN. \emph{Linear matrix inequalities in system and control theory}. SIAM, 1994.
	\bibitem{arz08} Denis ARZELIER. \emph{Course on LMI optimization with applications in control : part II.1 and II.2}. LAAS, 2008.
	\bibitem{dic12} Svetlana DICHEVA. \emph{Planification de mission pour un système de lancement aéroporté autonome}. Université d’Evry-Val d’Essonne, 2012.
\end{thebibliography}


