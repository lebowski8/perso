

\section{Cadre de travail à l'ONERA}
\subsection{Une entreprise historique}
L'ONERA - Office National d'Études et de Recherches Aérospatiales - est un organisme public de recherche pour les technologies aérospatiales. Il a été créé en 1946 pour accompagner le redémarrage de l'industrie aéronautique ravagée par la guerre. Depuis cette date, l'ONERA a contribué à tous les grands programmes aérospatiaux nationaux (Mirage, Concorde, fusées Diamant...) ou européens (Airbus, Ariane...). Son statut d'EPIC - Établissement Public à caractère Industriel et Commercial - lui assure une subvention étatique représentant environ 40\% de son budget annuel. Le reste provient d'une activité contractuelle, soit en son nom propre, soit au sein de consortium, pour les donneurs d'ordres variés (DGA, CNES, DASSAULT, ESA, UE, ANR...).

Ces travaux de recherche servent l'innovation et la compétitivité dans les secteurs de l'aéronautique, de l'espace et de la défense. Les missions de l'office sont de : 
\begin{itemize}
	\item Anticiper les ruptures technologiques pour \textbf{préparer l'avenir}
	\item Favoriser les \textbf{transferts vers l'industrie}
	\item Réaliser et mettre en œuvre des \textbf{moyens d'expérimentation et de simulation}
	\item \textbf{Fournir à l'industrie} des expertises de haut niveau
	\item \textbf{Expertiser pour l'État} les grands choix technologiques de demain
	\item \textbf{Former} des ingénieurs et des chercheurs
\end{itemize}

L'ONERA compte un peu plus de 2000 employés dont les trois quart sont des ingénieurs, techniciens ou doctorants. Son budget annuel dépasse les 220 millions d'euros.

Depuis Janvier 2007, pour répondre à une nécessité de visibilité internationale, le centre de recherche adopte la dénomination "ONERA : The French Aerospace Lab"

\subsection{La recherche à l'ONERA}
\textit{\textbf{La pluridisciplinarité des équipes}}

Les activités scientifiques sont organisées en 17 départements répartis en 4 branches scientifiques : 
\begin{itemize}
\item MFE : Mécanique des Fluides et Énergétiques
\item MAS : MAtériaux et Structures
\item PHY : PHYsique
\item TIS : Traitement de l'Information et Systèmes
\end{itemize}

L'ONERA est donc présent dans la plupart des disciplines scientifiques nécessaires aux progrès du secteur aérospatial. %En ce sens, il est déjà un organisme pluridisciplinaire. Mais il y a plus : ses chercheurs pratiquent au quotidien la fertilisation croisée des connaissances, en interne comme en externe. En réponse aux besoins de l'industrie, ils ont appris à travailler entre spécialistes de plusieurs disciplines, dans le cadre de projets à visées applicatives. Que ce soit entre équipes de l'ONERA, mais aussi avec d'autres, issues de laboratoires extérieurs.

\textit{\textbf{La dualité calcul-expérience}}

Les travaux de l'ONERA reposent sur une double approche : la simulation numérique et l'expérimentation. L'ONERA développe des codes de simulation numérique dans les différents domaines de l'aérospatial. Pour autant, l'expérimentation reste omniprésente, à chaque étape du processus d'acquisition des connaissances. Elle en est souvent le point de départ, sur des composants élémentaires dont on caractérise expérimentalement le comportement. Elle intervient ensuite, plus en aval, pour valider les modèles et recaler leurs paramètres sur des grandeurs physiques réelles.
 
\textit{\textbf{La connaissance des applications industrielles}}

En soixante-dix années de recherches menées pour le compte de l'industrie aéronautique, spatiale et de défense, l'ONERA a développé une tradition de confrontation avec l'application. Ses équipes ont tissé des liens avec les industriels et engrangé une somme inestimable de connaissances sur l'utilisation concrète de la science. Au final, les chercheurs sont en position d'anticiper les besoins de l'industrie et de proposer des solutions opérationnelles tenant compte de l'environnement applicatif.

\textit{\textbf{Un parc de moyens d'essais unique en Europe}}

%Pour mener sa double démarche Calcul/Expérimentation, l'ONERA dispose de trois catégories de moyens d'essais : premièrement les moyens d'essais de laboratoire

Pour mener sa double démarche Calcul/Expérimentation, l'ONERA dispose de trois catégories de moyens d'essais : premièrement les moyens d'essais de laboratoire comme les bancs de vélocimétrie laser, le canon à électrons... On trouvera également les moyens d'essais intermédiaires, c'est le cas de la maquette B20 à Lille, le banc de turbomachines aéronautiques Turma à Modane-Avrieux... Enfin les grandes souffleries dites "industrielles" du Fauga-Mauzac et de Modane-Avrieux constituent les grands moyens d'essais. Elles sont mises en œuvre, pour le compte de l'ONERA et pour des clients extérieurs, par une direction spécifique. Le parc de soufflerie de l'ONERA est le plus grand d'Europe.

\textit{\textbf{La dualité civil-militaire}}

L'activité de l'ONERA se répartit de façon équilibrée entre :
\begin{itemize}
	\item le civil
	\item la défense
	\item les recherches duales
\end{itemize}
Il est fréquent que des travaux initiés pour un secteur servent également à un autre par la suite.

\subsection{Départements d'accueil}

Mon stage s'est déroulé à l'ONERA Toulouse, dans le département DCSD - Commande des systèmes et dynamique du vol - de la branche TIS.

Les principaux champs d'actions du DCSD sont : 
\begin{itemize}
	\item Automatique
	\item Intelligence artificielle
	\item Robotique
	\item Conception et performances des systèmes aérospatiaux, cockpits et stations de contrôle
\end{itemize}

Le DCSD est lui même structuré en différentes unités de recherche. Mon stage s'est effectué en collaboration avec les unités CD - Conduite et Décision - pour ce qui est de la partie planification et automatique discrète, et CDIN - CommanDe et INtégration - pour ce qui est de la partie automatique continue et hybride.